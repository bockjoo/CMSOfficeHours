\documentclass[11pt]{article}

\usepackage{amsmath,amssymb,amsfonts}
\usepackage{graphicx}
%\usepackage{feynman}
\usepackage{bm}

\setlength{\topmargin}{-.5in} \setlength{\textheight}{9.25in}
\setlength{\oddsidemargin}{0in} \setlength{\textwidth}{6.8in}
\begin{document}
\Large
\noindent\textbf{CMS Office Hours} \\
\noindent\textbf{Quiz 1 Solutions: \emph{Relatively} Easy \hfill 2020-Jun-30}
\medskip\hrule
%%%%%%%%%%%%%%%%%%
\vspace{2mm}
\centering * * * Use natural units. $\hbar = c \equiv 1$ * * * \\
\centering A \textbf{Toolkit Equation} is one which you should 
add to your Physics Toolkit. \\
\textit{(...once you discover it of course!)}
\begin{enumerate}
\item 
    \begin{enumerate}
    \item Given that $E \geq 2m$ and using the \textbf{Toolkit Equation}: 
    $E = \gamma m$ (for a massive particle), we see that $\gamma \geq 2$.
    
    \item By definition,
    \begin{align}
        \gamma &\equiv \frac{1}{\sqrt{1 - \beta ^2}} \label{eqn:gammadef}
        , \;{\rm where\;} \beta \equiv v/c \\
        \implies \beta &= \sqrt{1 - \frac{1}{\gamma^2}}. \nonumber 
    \end{align}
    Using $\gamma = 2$ from before gives $\beta = 0.866$. 
    Therefore a particle becomes relativistic when it is going 86.6\% the speed of light! 

    \item Using $\beta = v/c$ and knowing that $0 \leq v \leq c$, it must be the case that 
    $$0 \leq \beta \leq 1.$$
    % \label{toolkit:Egammamu}
    From Equation \ref{eqn:gammadef} above, we see that $$1 \leq \gamma < \inf .$$

    \item
    %%%%%%%%%%%%%%%%%%%%
    \begin{figure}[b]
    \centering
    \includegraphics[width=9cm,height=9cm,keepaspectratio]{../Plots/plot_gamma_of_beta.pdf}
        \caption{Plot showing $\gamma$ as a function of $\beta$.} 
    \end{figure}
    %%%%%%%%%%%%%%%%%%%%

    \end{enumerate}
    
\pagebreak
% \item Calculate the Lorentz factor $(\gamma)$ of a proton moving at: 
%     \begin{enumerate}
%     \item 10\% the speed of light $(\beta \equiv \frac{v}{c} = 0.10)$. 
%     \item 90\% the speed of light.
%     \end{enumerate}

% Problem 2.
\item 
    \begin{enumerate}
        % \item Calculate its Lorentz factor.
        % \item Calculate its $\beta$ (where $\beta \equiv v/c)$.

        % Problem 2a.
        \item \emph{Look it up!} The lifetime of a muon ($\tau_{\mu}$) is about \textbf{2.2 $\bm{\mu}$s}. 
        This is how long it takes for the muon to decay in its rest frame. 

        % Problem 2b.
        \item The physicists in the lab will `see' the muon moving very fast; 
        its lifetime \emph{in the lab frame} will be \textbf{dilated}:
        \begin{align*}
            t_\mathrm{lab} &= \gamma \tau \\
            \gamma &= \frac{E}{m} 
            = \frac{\mathrm{60 \;GeV}}{\mathrm{0.1057 \;GeV}}
            = 567.6 \;\;({\textit ultra\;relativistic!}) \\ \\
            \implies t_\mathrm{lab} &= (567.6) (2.2\;\mathrm{\mu s}) = 1.25 \mathrm{\;ms}
        \end{align*}
        {\centering \textbf{Comment:} In this way, we can ``make particles live longer''
        by speeding them up! 
        The muon lives 500 times longer in the lab frame compared to
        its rest frame.
        }

        %%%%%%%%%%%%%%%%%%%%
        \begin{figure}[pt]
        \centering
        \includegraphics[width=15cm,height=15cm,keepaspectratio]{//Users/Jake/Desktop/Research/Figures/CMS/CMS_longitudinal_view_phase2.png}
            \caption{A quadrant view of the CMS detector.} 
        \label{fig:longitude_CMS}
        \end{figure}
        %%%%%%%%%%%%%%%%%%%%

        % Problem 2c.
        \item 
        In the lab frame the muon lives for 1.25 ms while travelling at nearly
        the speed of light:
        $$d_\mathrm{lab} = v_\mathrm{lab} t_\mathrm{lab} \approx c t_\mathrm{lab} = 375 \mathrm{\;Km}$$
        The distance from the interaction point 
        (i.e., the point where the protons collide) to the end of the Muon System is 
        about 7.5 m along the $\eta = 0$ direction, 
        about 11.0 m along the $\eta = 2.5$ direction, and therefore a maximum of
        about 13.3 m along the $\eta = 1.2$ direction 
        (convince yourself of this using Figure~\ref{fig:longitude_CMS}).
        
        Answer: \emph{Nope!} Physicists won't have to lose any sleep over 
        muons decaying before the muon system.

         % Problem 2d
        \item We need to relate the transverse momentum $(\vec{p}_{T})$ to the 
        charge of the particle $(q)$, magnetic field strength 
        $(\vec{B})$, and the
        radius of curvature $(R)$:
        \begin{equation}
            \label{eqn:darins_fav}
            \left| \vec{p}_{T} \right| = q \lvert \vec{B} \rvert R.
            \;\;\;\; \textbf{(Toolkit Equation)}
        \end{equation}
        You can convince yourself of this equation by using elementary classical mechanics
        of centripetal forces (see Fig.~\ref{fig:circular_motion}).
        Since $\gamma \approx 568 \gg 2$ and the motion of the muon is perpendicular to the $\vec{B}$ field, 
        we can say that 
        $$E = 60 \mathrm{\;GeV} \approx \vec{p} = \vec{p}_{T}.$$
        
        Since natural units take some getting used to, 
        let's use SI units in Eqn.~\ref{eqn:darins_fav} keeping in mind that
        $e = 1.602 \times 10^{-19}$ C:

        \begin{equation*}
            R = \frac{\left| \vec{p}_{T} \right|}{q \lvert \vec{B} \rvert} 
            = \frac{60 \mathrm{\;GeV}/c}{e (3.8 \mathrm{T})} \times
            \left( \frac{10^{9} \mathrm{\;eV}}{1 \mathrm{\;GeV}} \right)
            \left( \frac{e \mathrm{\;J/C}}{1 \mathrm{\;eV}} \right),
        \end{equation*}

        where the last conversion factor uses the fact that 
        $1.602 \times 10^{-19} \mathrm{\;J} = 1$ eV.
        Once the dust settles, we have:

        \begin{equation*}
            R = \frac{60 \times 10^{9} \mathrm{\;J}}
            {3 \times 10^{8} \mathrm{\;m/s}\cdot 3.8 \mathrm{\;T}} = 52.6 \mathrm{\;m}.
        \end{equation*}

        {\centering \textbf{Comment:} 
        In practice, this problem is worked in the \textit{reverse} way: 
        we measure the radius of curvature (technically the \textbf{sagitta})
        to figure out what the $p_{T}$ was.
        }
        %%%%%%%%%%%%%%%%%%%%
        \begin{figure}[bhtp]
            \centering
            \includegraphics[width=10cm,height=10cm,keepaspectratio]{//Users/Jake/Desktop/Research/Higgs_Mass_Measurement/d0_studies/Figures/figure_muon_in_B_field.png}
                \caption{A muon has a circular trajectory in a magnetic field.} 
            \label{fig:circular_motion}
            \end{figure}
        %%%%%%%%%%%%%%%%%%%%

        % Problem 2e
        \item 
        $$\mu^{+} \to \bar{\nu}_{\mu} + e^{+} + \nu_{e}$$ \\
        {\centering \textbf{Comment:} 
        The positron is the only particle with a mass \textit{lighter} than the muon
        and which also conserves its electric charge ($+1$). 
        The 2 neutrinos are there to conserve 
        \textbf{muon lepton number} ($-1$ on both sides) and 
        \textbf{electron lepton number} (0 on both sides). 
        \textit{The neutrinos are also necessary to conserve momentum!}
        }

        % Problem 2f
        \item 
        $$\mu^{-} \to \nu_{\mu} + e^{-} + \bar{\nu}_{e}$$
        
    \end{enumerate}

% Problem 3.
\item In High-Energy Particle Physics, there are a couple useful \textit{Pythagorean equations} 
(equations of the form: $c^2 = a^2 + b^2$).
    \begin{enumerate}
    % Problem 3a.
    \item 
    $$E^2 = m^2 + {\vec{p}}\;^2 \;\;\;\;{\textbf{(Toolkit Equation)}}$$
    % Problem 3b.
    \item Starting from Eqn. \ref{eqn:gammadef}:
    
    $$\gamma \equiv \frac{1}{\sqrt{1 - \beta ^2}}$$
    Do some algebra to get:
    $$1^2 = \beta^{2} + \frac{1}{\gamma^{2}} \;\;\;\;{\textbf{(Toolkit Equation)}}$$
    {\centering \textbf{Comment:} 
        Often during special relativity calculations, you will come across
        factors like: $1/\gamma^2, \gamma^2 \beta^2$, etc. so it is useful
        to have this equation handy to convert easily between them.
    }
    \end{enumerate}
\end{enumerate}

\end{document} 